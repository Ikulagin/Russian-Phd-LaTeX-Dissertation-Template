\chapter{Оптимизация выполнения программ на многоядерных SMP/NUMA системах} \label{chapt3}

\section{Эксперименты}
Экспериментальное исследование проводилось на вычислительной системе, оснащенной двумя четырехъядерными процессорами Intel Xeon E5420. В данных процессорах отсутствует поддержка аппаратной транзакционной памяти (Intel TSX). В качестве тестовых программ использовались многопоточные STM-программы из пакета STAMP \cite{}. Число потоков варьировалось от 1 до 8. Тесты собирались компилятором GCC 5.1.1. Операционная система GNU/Linux Fedora 21 x86\_64. 

В рамках экспериментов измерялись значения двух показателей:
\begin{itemize}
\item время $t$ выполнения STM-программы;
\item количество $C$ ложных конфликтов в программе.
\end{itemize}

На рис.~\ref{fc_S_const}, ~\ref{fc_B_const}, ~\ref{time_B_const}, ~\ref{time_B_const} показана зависимость количества $C$ ложных конфликтов и времени $t$ выполнения теста от числа потоков при различных значениях параметров $B$ и $S$. Результаты приведены для программы genome из пакета STAMP. В ней порядка 10 транзакционных секций, реализующих операции над хеш-таблицей и связными списками. Видно, что увеличение значений параметров $S$ и $B$ приводит к уменьшению числа возможных коллизий (ложных конфликтов), возникающих при отображении адресов линейного адресного пространства процесса на записи таблицы. При размере таблицы $2^{21}$ записей, на каждую из которых отображается $2^{6}$ адресов линейного адресного пространства, достигается минимум времени выполнения теста genome, а также минимум числа ложных конфликтов. 

\begin{figure}
\centering
\begin{subfigure}{0.5\textwidth}
  \centering
  \includegraphics[width=0.8\linewidth]{fc_S_const}
  \caption{$S = 2^{19}$}
  \label{fc_S_const}
\end{subfigure}%
\begin{subfigure}{0.5\textwidth}
  \centering
  \includegraphics[width=0.8\linewidth]{fc_B_const}
  \caption{$B = 2^6$}
  \label{fc_B_const}
\end{subfigure}
\caption{Зависимость числа $C$ ложных конфликтов от числа $N$ потоков}
\label{fc_N}
\end{figure}

\begin{figure}
\centering
\begin{subfigure}{0.5\textwidth}
  \centering
  \includegraphics[width=0.8\linewidth]{time_S_const}
  \caption{$S = 2^{19}$}
  \label{time_S_const}
\end{subfigure}%
\begin{subfigure}{0.5\textwidth}
  \centering
  \includegraphics[width=0.8\linewidth]{time_B_const}
  \caption{$B = 2^6$}
  \label{time_B_const}
\end{subfigure}
\caption{Зависимость времени $t$ выполнения теста от числа $N$ потоков}
\label{time_N}
\end{figure}

Время выполнения теста genome удалось сократить в среднем на 20\% за счет минимизации числа ложных конфликтов.

\section{Выводы}

%\newpage
%============================================================================================================================

% \section{Параграф - два} \label{sect3_2}

% Некоторый текст.

%\newpage
%============================================================================================================================

% \section{Параграф с подпараграфами} \label{sect3_3}

%\subsection{Подпараграф - один} \label{subsect3_3_1}

%Некоторый текст.

%\subsection{Подпараграф - два} \label{subsect3_3_2}

%Некоторый текст.

\clearpage


